\documentclass[11pt,a4paper]{article}
\usepackage[utf8]{inputenc}
\usepackage[spanish]{babel}
\usepackage{amsmath}
\usepackage{amsfonts}
\usepackage{amssymb}
\usepackage{cancel}
\usepackage{graphicx}
\usepackage[usenames,dvipsnames,svgnames,table]{xcolor}
\usepackage[left=2cm,right=2cm,top=2cm,bottom=2cm]{geometry}
\usepackage{hyperref}
\usepackage{caption}
\usepackage{subcaption}
\author{Rodrigo Vega Vilchis\\
Correo: rockdrigo6@ciencias.unam.mx\\
\small Medios deformables}
\title{Notas Medios deformables}
\date{Semestre 2022-1}


\begin{document}
\maketitle

\vspace*{-1.3cm}
\setlength{\unitlength}{1cm}

\begin{picture}(10,0) 
\put(0,2.5){\includegraphics[scale=.15]{unam1.png}}
\put(14.7,2.4){\includegraphics[scale=.22]{ciencias.png}}  
\end{picture}

\begin{abstract}
Estas notas están dedicadas al curso de Dinámica de medio deformables con la Profesora Catalina Stern. El contenido de las mismas será exclusivamente de las clases de Catalina o que consdiere como información relevante, ya que existen ayudantías (por fortuna) en donde revisamos ejercicios que no son tan urgentes para incluirlos aquí.
\end{abstract}

\section{Medios continuos}

Los medios continuos son materiales que llenan todo el espacio sin poros. Las propiedades de un medio continuo pueden representarse por medio de funciones continuas de clase $C^1$, es decir, con primeras derivadas continuas.\footnote{Checar si no deben ser de tipo $C^\infty$}. Nos manejamos en medios homogéneos: que son aquellos cuyas propiedades son las mismas en cada punto del espacio; medios isotrópicos: si las propiedades son las mismas en todas direcciones.

En el curso se trabajarán principalmente con \emph{principios generales}, \emph{ecuaciones constitutivas}\footnote{Son todas aquellas que se determinaron durante el periodo de la física clásica, las de mecánica clásica de Newton, termodinámica de Clausius-Kelvin y electromagnetismo de Maxwell}, y sobre todo consideraremos \emph{casos ideales} a temperatura constante y sin intervenciones electromagnéticas (todo muy sencillito).

Antes de pasar a lo bueno, vamos a escribir un repaso con los elementos necesarios para llevar el curso de manera amena.

\subsection{Repaso}

\subsubsection{Vectores y Tensores}

En general hay una nota de que estas clases están basadas en el Malvern y el Reddy. Como la descripción de fenómenos debe ser independiente de la posición y orientación del sistema de referencia, utilizamos \textbf{vectores} y \textbf{tensores} que se transforman fácilmente de un sistema de referencia a otro.

De aquí en adelante iré poniendo lo más relevante que se ha visto. Por ejemplo, es conveniente ver a un vector en $\mathbb{R}^3$ como combinación lineal de cada una de sus componentes:
\begin{align*}
\vec{a}&=\vec{a}_x+\vec{a}_y+\vec{a}_z\\
&=a_x\vec{e}_x+a_y\vec{e}_y+a_z\vec{e}_z
\end{align*}
donde las $a_i$ son escalares \textbf{reales} y los $\vec{e}_i$ son vectores canónicos de $\mathbb{R}^3$, $i\in\{x,y,z\}$. Podemos definir cada escalar $a_i$ con base en el ángulo de inclinación que hace con respecto de cada eje coordenado. Es decir
\begin{align*}
a_x&=a\cos\alpha_x\\
a_y&=a\cos\alpha_y\\
a_z&=a\cos\alpha_z
\end{align*}
donde $\|\vec{a}\|=a$ y los ángulos $\alpha_i$ con $i\in\{x,y,z\}$ son los ángulos de inclinación para cada eje. Quizás valga la pena recordar este truquito, si sacamos la norma cuadrada del vector $\vec{a}$ y consideramos a los cosenos directores tenemos
\begin{align*}
\|\vec{a}\|^2=a^2&=a_x^2+a_y^2+a_z^2\\
&=a^2\left(\cos^2\alpha_x+\cos^2\alpha_y+\cos^2\alpha_z\right)\\
1&=\cos^2\alpha_x+\cos^2\alpha_y+\cos^2\alpha_z
\end{align*}
Sabemos bien que como hablamos de $\mathbb{R}^3$\footnote{no se si vayamos a usar otro espacio vectorial} es claramente un espacio vectorial sobre $\mathbb{R}$, y tiene toda su estructura, por lo que obviaré las definiciones.

\subsubsection{Convención de la suma de Einstein}

De la norma cuadrada\footnote{vamos a considerar el abuso de notación $\|\vec{a}\|^2=a^2$} sabemos hemos visto bien como se escribe y la podemos escribir en términos de una suma, tal y como sigue
\begin{align*}
a^2=a_1^2+a_2^2+a_3^2&=\sum_{i=1}^3a_i^2\\
&=a_ia_i
\end{align*}
vamos a intercambiar las letras de cada eje coordenado por números (es equivalente, solo es notación). Tenemos que distinguir que tenemos dos índices que se repiten, esto es lo que caracteriza la \textbf{Convecnión de la suma de Einstein}. Debemos distinguir también que 
$$a_ia_i\neq a_i^2$$
del lado izquierdo tenemos indices repetidos y por ello consideramos una suma, del lado derecho solo estamos diciendo que el término “$a_i$'' esta elevado al cuadrado. Los índices repetidos se les denomina \textbf{índices mudos}: e indican una suma. Los \textbf{índices libres} son los que no se repiten y no representan una suma ni nada.

Hablemos ahora de las “rotaciones'': decimos que las rotaciones finitas NO son vectores porque no son conmutativas, pero las rotaciones infinitesimales si son vectores, hablemos del \textbf{producto escalar}, rápidamente definido como
$$\vec{a}\cdot\vec{b}=\|a\|\|b\|\cos\theta,\qquad\text{con $\theta$ el ángulo entre $\vec{a}$ y $\vec{b}$}$$
igualmente, no ahondaré en las propiedades de producto escalar (por el tiempo que tengo actualmente) bien podemos revisarla en los libros de Álgebra Lineal. Definamos ahora el producto vectorial como
\begin{align*}
\vec{c}&=\vec{a}\times\vec{b}\\
&=\|a\|\|b\|\sin\theta,\qquad\text{$\theta$ igual que en el caso anterior.}
\end{align*}
vale la pena recordar que el producto vectorial genera un vector ortogonal a los vectores participantes en cuestión, ($\vec{a}$ y $\vec{b}$) y también precisar que posee la propiedad de \emph{anticonmutatividad}. Al producto vectorial lo vemos con el siguiente determinante
$$\vec{a}\times\vec{b}=\begin{vmatrix}
\vec{e}_x & \vec{e}_y & \vec{e}_z\\
a_1 & a_2 & a_3\\
b_1 & b_2 & b_3\\
\end{vmatrix}=(a_2b_3-b_2a_3)\vec{e}_x+(a_1b_3-b_1a_3)\vec{e}_y+(a_1b_2-b_1a_2)\vec{e}_z$$
podemos hacer un tratamiento de estos productos por medio de convención de suma de Einstein, pero antes definamos los símbolos de Levi-Civita (o de permutación)

\subsubsection{Símbolos de Levi-Civita (o de permutación)}

Lo definimos como 
$$\varepsilon_{mnr}=\begin{cases}
0,\qquad &\text{si hay índices repetidos}\\
1, &\text{orden positivo de permutaciones}\\
-1, &\text{oden inverso de permutaciones}
\end{cases}$$
Ejemplos de orden positivo e inverso de permutaciones:
\begin{align*}
\varepsilon_{123}&=\varepsilon_{231}=\varepsilon_{312}=1,\qquad \text{las permutaciones van hacia la derecha}\\
\varepsilon_{213}&=\varepsilon_{321}=\varepsilon_{132}=-1\qquad \text{las permutaciones van hacia la izquierda}
\varepsilon_{223}=0
\end{align*}
definamos ahora la delta de Kronecker:
$$\delta_{ij}=\begin{cases}
1,\qquad i=j\\
0,\qquad i\neq j
\end{cases}$$
Podemos relacionar ámbos símbolos por medio de un producto y una resta de productos de la siguiente manera
$$\varepsilon_{ijk}\varepsilon_{irs}=\delta_{jr}\delta_{ks}-\delta_{js}\delta_{kr}$$
en este caso, tenemos que $i$ es mudo y los otros son libres, entonces haciendo los cálculos, esto implica que
$$\sum_{i=1}^{3}\varepsilon_{i23}\varepsilon_{i12}=\varepsilon_{i23}\varepsilon_{i12},\qquad\text{sumamos sobre }i$$
el producto vectorial como lo vimos anteriormente lo podemos escribir en términos de los símbolos de permutación como
\begin{align*}
c_p&=\varepsilon_{pqr}a_pb_r\\
c_3&=\varepsilon_{321}\ a_1b_2+\varepsilon_{321}\ a_2b_1\\
&=a_1b_2-a_2b_1
\end{align*}
haciendo los cálculos correspondientes, claro. Veamos ahora como podemos hacer \emph{cambios} de sistema de referencia, veremos como actúa la matriz de transformación o \textbf{tensor métrico}

\subsubsection{Sistemas de referencia}

Podemos tener un sistema de referencia con la base 
\begin{align*}
E_1&=\{\vec{e}_i\},\qquad i\in\{1,2,3\}\\
E_2&=\{\vec{e}_i\, '\},\qquad i\in\{1,2,3\}\\
\end{align*}
podemos expresar un vector arbitrario $\vec{a}$ en ambos sistemas de referencia, tal y como sigue
\begin{align*}
\vec{a}&=a_i\vec{e}_i=(\vec{a}\cdot\vec{e}_i)\vec{e}_i\\
&=a_j'\vec{e}_j\, '=\underbrace{(\vec{a}\cdot\vec{e}_j\, ')}_{a_j'}\vec{e}_j\, '
\end{align*}
notemos que en la segunda igualdad estamos proyectando el vector $\vec{a}=a_i\vec{e}_i$ con respecto del vector $\vec{e}_j\, '$, fijémonos como deben ser los escalares $a_j'$ y sobre todo ver qué resulta de ello
$$a_j'=\vec{a}\cdot\vec{e}_j\, '=(a_i\vec{e}_i)\cdot\vec{e}_j\, '=a_i(\vec{e}_i\cdot\vec{e}_j\, ')=\ell_{ij}a_i$$
donde $\ell_{ij}=\vec{e}_i\cdot\vec{e}_j\, '$. Definimos ahora la matríz de transformación o \textbf{tensor métrico} como 
$$L=(\ell_{ij}),\qquad \text{en este caso, la $i$ representa el sistema nuevo y la $j$ el sistema original}$$
¿Cuál es el sentido físico de ver transformaciones lineales de vectores? Al deformar una cantidad física representada por un vector debido a una fuerza: un vector cualquiera $\vec{u}$ se convierte en $\vec{v}$ en el mismo sistema de referencia
$$\vec{v}=T\vec{u}$$
donde $T$ es una matriz de transformación o tensor métrico.

\subsubsection{Ejemplos y Resultados importantes}

En esta sección veremos algunos resultados vistos en la primera ayudantía del curso. En dicha ayudantía estuvimos viendo ejercicios de como operar con los productos escalares, productos vectoriales con base en las deltas de Kronecker y los símbolos de permutación.

Enumeraré a continuación los resultados que considere importantes
\begin{enumerate}
\item Ejercicio no. 3
\begin{align*}
(\vec{a}\times\vec{b})\cdot(\vec{c}\times\vec{d})&=a_ib_jc_id_j-a_mb_rc_rd_m\\
&=(\vec{a}\cdot\vec{c})(\vec{b}\cdot\vec{d})-(\vec{a}\cdot\vec{d})(\vec{b}\cdot\vec{c})
\end{align*}
\item Gradiente de un campo escalar
$$\nabla f=\vec{e}_i\frac{\partial}{\partial x_i}f$$
\item Divergencia de un campo vectorial
$$\nabla\cdot \vec{F}=(\vec{e}_i\partial_i)\cdot v_j\vec{e}_j\ \footnote{De ahora en adelante, vamos a denotar las derivadas parciales como $\partial_i=\frac{\partial}{\partial x_i}$}$$
\item Laplaciano 
\begin{align*}
\nabla^2 f=\nabla\cdot\nabla f&=(\vec{e}_i\partial_i)\cdot(\vec{e}_j\partial_j f)\\
&=\partial_i\partial_j f\delta_{ij}\\
&=\partial_i\partial_i f
\end{align*}
\item Rotacional
\begin{align*}
\nabla\times \vec{F}&=(\vec{e}_i\partial_i)\times(F\vec{e}_i )\\
&=\partial_i v_j\vec{e}_i\times\vec{e}_j\\
&=\partial_i v_j\ \varepsilon_{ijk}\vec{e}_k
\end{align*}
\textbf{Resultados curiosos:}
\item $\nabla\times\nabla\times\vec{F}$
\begin{align*}
\nabla\times\nabla\times\vec{F}&=\nabla(\nabla\cdot \vec{F})-\nabla^2\vec{F}\\
&=\partial_i\partial_j F_i\vec{e}_j-\partial_i\partial_i F_k\vec{e}_k
\end{align*}
\end{enumerate}

Continuaremos en la siguiente sección con la ayudantía del 27 de Septiembre 2021.

\end{document}